1. The forEach() method:
Arrays in JavaScript come with a handy method that allows you to loop over each of their members. 

Example:
const fruits = ['kiwi','mango','apple','pear'];
function appendIndex(fruit, index) {
    console.log(`${index}. ${fruit}`)
}
fruits.forEach(appendIndex);

Output: 

0. kiwi
1. mango
2. apple
3. pear

The forEach() method accepts a function that will work on each array item. 
That function's first parameter is the current array item itself, 
and the second (optional) parameter is the index.

The function that the forEach() method needs to use is 
passed in directly into the method call, like this:


const veggies = ['onion', 'garlic', 'potato'];
veggies.forEach( function(veggie, index) {
    console.log(`${index}. ${fruit}`);
});

This makes for more compact code, but perhaps somewhat harder to read. 
To increase readability, sometimes arrow functions are used.